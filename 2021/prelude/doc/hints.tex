\documentclass[12pt]{scrartcl}
\usepackage{amssymb,graphicx}

\title{An atonal challenge}
\subtitle{Ph0wn 2021}

\date{}

\begin{document}
\maketitle


\section*{Pitch Classes and Set Theory}

Pitch-class Set Theory is field of musical analysis often applied for analyzing atonal music. Pitch-class Set Theory studies sets of \emph{pitch-classes}.

A pitch-class gathers all the notes that are $n$ octaves apart, with $n \in \mathbb{N}$. For instance, the pitch-class of C will gather C1, C2, C3, etc.

There are 12 different pitch-classes, and each of them are associated with an integer $i \in [\![0,11]\!]$. Table~\ref{table:pitch-classes} provides the equivalence between pitch classes and $[\![0,11]\!]$.

\begin{table}[ht]
	\centering
	\begin{tabular}{c|c}
		$i$ & Pitch-class\\
		\hline
		\hline
		0 & C\\
		\hline
		1 & C\#, D$\flat$\\
		\hline
		2 & D\\
		\hline
		3 & D\#, E$\flat$\\
		\hline
		4 & E\\
		\hline
		5 & F\\
		\hline
		6 & F\#, G$\flat$\\
		\hline
		7 & G\\
		\hline
		8 & G\#, A$\flat$\\
		\hline
		9 & A\\
		\hline
		10 & A\#, B$\flat$\\
		\hline
		11 & B
	\end{tabular}
	\caption{The 12 pitch-classes.}
	\label{table:pitch-classes}
\end{table}

See Wikipedia ``Pitch Class'' for more information.

\section*{Chord}

A chord is a group of 1 or more notes played at the same time. So, it converts to an integer set.

Example:
\begin{itemize}
  \item The C-Major chord (C, E, G) corresponds to the set $\{0,4,7\}$.
  \item Let's suppose we have a C3-C4-E4-G4-C5 chord, the pitch-classes composing this chord are the C-pitch-class, the E-pitch-class and the G-pitch-class, therefore the corresponding set will be $\{0,4,7\}$.
  \item Reciprocally, a single note C3 corresponds to $\{0\}$.
\end{itemize}



\end{document}
